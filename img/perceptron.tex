\documentclass{standalone}

\begin{document}
  \def\nodesize{1.0cm}
  \tikzstyle{neuron}=[circle, draw=black, minimum size=\nodesize]
  \tikzstyle{dots}=[draw=none, scale=2, text height=0.333cm, execute at end node=$\vdots$]

  \begin{figure}[!htp] %place it here (h)
    \centering
    \begin{tikzpicture}[x=1.5cm, y=1.5cm, >=stealth] % >=stealth arrow style
      % input nodes
      \foreach \m [count=\y] in {1,2,3,dots,n} 
      {
        \ifnum\y=4
          \node [\m/.try] at (0,-\y) {};
        \else
          \node [neuron] (input-\m) at (0,-\y) {$x_\m$};
        \fi
      }
      % middle node
      \node [neuron] (middle) at (1.5, -3) {\LARGE $\Sigma$};

      % activation function
      \node [neuron] (activation) at (3.0, -3) {$g$};

      % connections
      \foreach \m [count=\y] in {1,2,3,n}
        % above moves the text higher depending on the orientation of the line
        \draw [thick, ->] (input-\m) -- (middle) 
              node[above=4pt*abs(3 - \y), pos=0.45, scale=0.9] {$\theta_\m$};

      % middle point between (middle) and (activation) nodes
      \path [thick, ->, draw] (middle) -- (activation) coordinate[pos=0.5] (activation-path);


      % draw sigmoid symbol betwwen (middle) and (activation)
      \draw [thick, scale=0.3, color=funcolor, x=5, y=70, 
            domain=-6:6, yrange=-1:1, 
            shift={($ (activation-path) - (0, 0.5) $) }] % calc library 
              plot (\x, {1/(exp(-\x)+1)});

      % alternative representation of the sigmoid symbol using an integral
      % \draw [thick, ->] (middle) -- (activation) node[midway, above] {
      %   $\displaystyle \int$
      % };

    \end{tikzpicture}
    \caption{Schematizzazione del Perceptron}
    \label{fig:perceptron}
  \end{figure}
\end{document}
