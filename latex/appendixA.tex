\begin{document}
\chapter{Metriche di Errore}
Le metriche utilizzate per valutare l'efficacia del modello sviluppato sono:
\begin{itemize}
  \item RMSE (Root Mean Squared Error): indica l'errore quadratico medio
    commesso dal modello su un insiete di dati e sottomesso all'operazione di
    radice quadrata. Tale metrica pone particolare peso sulla entità
    dell'errore piuttosto che sul loro numero, per via dell'operazione di
    elevazione al quadrato. L'adozione della radice permette di valutare la
    metriche nell'unità di misura di riferimento; in questo caso il metro,
    poichè le coordinate dell'output del modello sono espresse in metri. Un
    basso valore di MSE può quindi comportare un alto numero di errori commessi
    a fronte di una alta precisione.
  \item MAE (Mean Absolute Error): indica l'errore assoluto medio commesso dal
    modello ed è utilizzato per valutare l'efficacia del modello ponendo meno
    attenzione sulla presenza di eventuali valori anomali. Come nel caso di
    RMSE, anche questa metrica esprime una valutazione della precisione del
    modello in metri.
  \item MaxAE (Max Absolute Error): determina il massimo errore assoluto
    commesso dal modello ed è utile per ottenere una stima superiore
    della precisione del modello. Un alto valore di MaxAE correlato con un
    basso valore di MAE indica una buona precisione media, ma con errori
    significativi e sporadici. Questa metrica, poichè non standardizzata, non è
    presente tra quelle di TensorFlow ed è stato necessario implementarla.
\end{itemize}

\end{document}
