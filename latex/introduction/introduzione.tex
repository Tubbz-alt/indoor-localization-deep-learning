\documentclass[draft]{standalone}
\begin{document}
%Problema
\section{Localizzazione Indoor}
Il problema della Localizzazione Indoor consiste nell'individuazione di un
utente all'interno di uno spazio chiuso, in riferimento a un sistema di
coordinate predefinito. Tale sistema di coordinate, relativo ad un determinato
edificio, può essere poi espresso in termini georeferenziali conoscendo la
precisa dislocazione geografica del locale in questione. 

La localizzazione indoor apre le porte a diverse possibilità nel campo
dell'esperienza utente all'interno di edifici pubblici, nel settore della
gestione dei flussi di persone, della sicurezza e della contingentazione. Tali
problematiche si fanno ancora più rilevanti a fronte della recente epidemia di
Covid-19 che ha colpito il pianeta. Attraverso l'impiego di tale tecnologia è
possibile coadiuvare la navigazione degli utenti all'interno di edifici
complessi, assicurare il rispetto delle norme di distanziamento sociale
interpersonale e migliorare l'esperienza individuale di persone affette da
disabilità. Per ottenere questi risultati è però richiesto un certo grado di
precisione, di affidabilità, di efficienza e di sicurezza nella gestione della
privacy dei dati di localizzazione degli utenti. Inoltre la tecnologia scelta
per risolvere il problema, per essere fruibile, deve avere come ulteriore
requisito il basso impatto economico.

\section{Soluzioni Tecnologiche}
%Letteratura
Nel corso degli anni sono state implementati diversi sistemi di localizzazione
indoor, che possiamo dividere in due macrocategorie: soluzioni ad-hoc e
soluzioni che sfruttano tecnologie esistenti. Nel primo caso si fornisce
all'utente l'attrezzatura necessaria ad essere localizzato, mentre nel secondo
si utilizza un dispositivo mobile di proprietà dell'utilizzatore.  Spesso tale
dispositivo è uno smartphone. 

I sistemi che implementano tecnologie sviluppate ad-hoc, sono spesso più
efficienti, più precisi e flessibili. Tuttavia, il loro impiego rimane limitato
a causa dell'alto costo di progettazione, installazione e di gestione. È poi
richiesto che ad ogni utente che intenda essere localizzato sia assegnato un
dispositivo che si interfacci col sistema impiegato.

Per l'impiego su larga scala, un sistema di localizzazione indoor deve essere
facilmente utilizzabile dalle masse e non deve richiedere particolari requisiti
tecnologici.

\section{Bluetooth Low Energy}
La tecnologia \emph{Bluetooth} è talmente pervasiva che ogni smartphone in
circolazione ne implementa il protocollo, mostrandosi particolarmente adeguata
alla risoluzione del problema in esame. Nello specifico, \emph{Bluetooth Low
  Energy} (BLE) è un protocollo che riduce notevolmente il consumo energetico
dei dispositivi che lo utilizzano. La tecnologia BLE viene utilizzata dalle
moderne API di \emph{contact tracing} sviluppate da Apple e Google, nonchè
dalla applicazione Immuni\cite{immuni} per il tracciamento dei contagi di
Covid-19\cite{apple-google}.

La soluzione riportata in questo elaborato prevede l'utilizzo di una serie di
beacon BLE programmabili, ciascuno installato in un punto significativo
dell'edificio e configurato per emettere un segnale broadcast con una frequenza
di circa 50Hz. La potenza dei segnali viene quindi utilizzata per produrre,
attraverso l'utilizzo di una rete neurale artificiale, una coppia di coordinate
rappresentative della posizione dell'utente all'interno dell'edificio. Ciò
viene reso possibile da una fase preliminare in cui viene mappata la superficie
del locale raccogliendo i segnali ricevuti dai beacon in vari punti. Per ogni
punto della superficie mappato si registra una serie temporale di segnali, dei
quali si considera solo il valore \emph{RSSI}, ovvero la potenza del segnale
nel punto in cui questo viene ricevuto.

Il modello utilizzato è di fatto completamente agnostico rispetto
all'ubicazione dei beacon installati, supponendo che questa sia unica e non
mutabile nel tempo.

L'utilizzo di tale sistema assicura il completo anonimato dell'utente, il quale
non necessita di condividere la propria posizione, essendo quest'ultima
calcolata direttamente sul suo smartphone in funzione dei segnali che riceve.

% Contenuto
Questo elaborato si pone l'obiettivo di descrivere nello specifico la rete
neurale progettata per risolvere il problema, le tecniche utilizzate per alzare
il grado di precisione del modello e le principali differenze rispetto a
modelli già esistenti.

\section{Sperimentazione e Collaborazioni}
Il progetto esposto in questo elaborato è stato realizzato per conto del
Consorzio Metis e la sperimentazione è stata eseguita presso i locali dell'ASL
Toscana Nord Ovest di Pisa. Il lavoro è stato svolto in autonomia, ma sia il
Consorzio Metis che l'ASL hanno predisposto i propri ambienti per
l'installazione dei Beacon Bluetooth e la raccolta dei dati.

% \section{Installazione dei Beacon e Acquisizione dei Dati}

\end{document}
