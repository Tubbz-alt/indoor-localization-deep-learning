\begin{document}
Questo capitolo introduce una descrizione dell'applicazione mobile sviluppata
come prototipo per il progetto di navigazione indoor realizzato per conto del
Consorzio Metis e ASL Toscana. I seguenti paragrafi analizzano le tecnologie
software usate, l'architettura dell'applicazione e i metodi utilizzati per
produrre una navigazione indoor fluida sulla base del modello di machine
learning prodotto precedentemente.
\section{Flutter}
L'applicazione mobile è stata sviluppata utilizzando 
Flutter\cite{flutter}, un framework per lo sviluppo cross-platform di
applicazioni. Tramite Flutter è possibile utilizzare una singola codebase per
produrre applicativi eseguibili nativamente su architetture desktop, smartphone
Android e iOS, e sul web. Il linguaggio di programmazione utilizzato da Flutter
è Dart, un linguaggio funzionale general purpose recentemente sviluppato da
Google.
%%%%%%%%%%%%%%%%%%%%%%%%%%%%%%%%%%%%%%%%%%%%%%%%%%%%%%%%%%%%%%%%%%%%%%%%%%%%%%%
\section{Planimetrie e Poligoni}
Per implementare la navigazione indoor, l'applicazione è stata dotata di un
visualizzatore di planimetrie. Queste ultime, prima di essere processate dal
componente grafico, vengono convertite in un insieme di poligoni ciasuno
indicante un vano o un elemento del locale da visualizzare. I poligoni vengono
poi disegnati a schermo in modo da mostrare la planimetria all'utente, insieme
alla sua presunta posizione all'interno dell'edificio. In
Figura~\ref{fig:floorplan} è mostrata l'interfaccia dell'applicazione mentre
visualizza la planimetria della sede del Consorzio Metis.
\begin{figure}[!htp]
  \centering\includegraphics[width=0.3\textwidth]{./img/floorplan.eps}
  \caption{Screenshot del visualizzatore di planimetrie: in alto sono presenti
  due slider che gestiscono rispettivamente il coefficiente di memoria residua
del modello e la frequenza di campionamento dei segnali}% 
  \label{fig:floorplan}%
\end{figure}
%%%%%%%%%%%%%%%%%%%%%%%%%%%%%%%%%%%%%%%%%%%%%%%%%%%%%%%%%%%%%%%%%%%%%%%%%%%%%%%
\section{Backend TensorFlow}
L'applicazione proposta sfrutta le librerie di TensorFlow per caricare in
memoria il modello di machine learning descritto nel
capitolo~\ref{cap:architecture}. Le stesse librerie permettono di utilizzare
tale modello per eseguire inferenze sui dati. 

Per ottimizzare le perfomance del modello su dispositivi mobile dotati di
ridotte capacità computazionali, il modello è stato prima compresso tramite
gli strumenti forniti da TensorFlow Lite\cite{tensorflow-lite}. Tensorflow Lite
permette di minimizzare lo spazio occupato in memoria del modello e
contemporaneamente di ottimizzare le operazioni in virgola mobile
quantizzandone gli operandi. Ciò comporta un'esecuzione a runtime più veloce e
tempi di latenza minori durante la navigazione, al costo di una precisione
inferiore che però non compromette significiativamente il risultato della
computazione.
% \subsection{Implementazione del Bridge di Comunicazione}
%%%%%%%%%%%%%%%%%%%%%%%%%%%%%%%%%%%%%%%%%%%%%%%%%%%%%%%%%%%%%%%%%%%%%%%%%%%%%%%
\section{Stabilizzazione del Modello}
Come descritto precedentemente, le fluttuazioni dell'output prodotto dal
modello potrebbero inficiare l'esperienza d'uso dell'applicazione da parte
dell'utente. Seppure la precisione media del modello sia molto bassa, errori
isolati e rapidi scostamenti dal valore atteso sono possibili e non
eliminabili. Di conseguenza si è reso necessario applicare alcune tecniche di
\emph{denoising} in modo da rendere la navigazione più fluida e costante. Tra
queste vi è l'utilizzo di un filtro di Kalman e dei sensori inerziali dello
smartphone.
\subsection{Utilizzo di Sensori Inerziali}
Tutti i moderni smartphone contengono al loro interno alcuni sensori inerziali,
i quali possono essere sfruttati per determinare la posizione dell'utente nello
spazio e il loro stato di moto. In particolare l'accelerometro è un sensore che
misura le variazioni dell'accelerazione subite dal dispositivo lungo i tre assi
dello spazio. Tali valori vengono collezionati periodicamente e possono essere
utilizzati per individuare la velocità dell'utente che si muove all'interno
dell'edificio. Poichè l'accelerazione è la derivata della velocità, le
componenti di quest'ultima possono essere ricavate dalla relazione:
\[v(t) = v_0 + \int_{t_0}^t{a(t) dt}. \]
Per calcolare numericamente il valore di tale integrale è stata utilizzata la
regola del trapezio, la quale permette di approssimare il valore di un
integrale definito dividendone l'intervallo in \(n\) sottonsiemi disgiunti e
calcolando per ciascuno di essi l'area del trapezio contenuto tra l'asse delle
ascisse e la funzione in quell'intervallo. La somma di queste aree parziali
rappresenta una approssimazione dell'integrale richiesto, la quale migliora al
crescere di \(n\). Denotando con \(\tilde{v}(t)\) l'approssimazione dell'area
dell'integrale dell'accelerazione ricavata dal sensore al tempo \(t\), si ha:
\[ \tilde{v}(t) = 
  v_0 + \frac{t - t_0}{n} \sum_{i=0}^{n-1} {\frac{a(t_i) - a(t_{i+1})}{2}}. \]

La precisione dei sensori inerziali degli smartphone attuali è tuttavia
piuttosto bassa ed è difficile ottenere una stima accurata della velocità
relativa del dispositivo tramite metodi numerici, i quali approssimano
solamente la soluzione corretta. Per questi motivi si è scelto di non
utilizzare direttamente il valore calcolato della velocità dell'utente o il suo
orientamento, ma di limitare la valutazione del suo stato di moto al calcolo di
un coefficiente indicativo della confidenza con cui si asserisce che l'utente
si stia muovendo. Tale coefficiente è definito come
\[ \phi = \tanh(|\tilde{v}|). \]
L'uso della funzione \(\tanh\) e del valore assoluto permette di proiettare il
codominio della funzione nell'intervallo \( (0,1) \). In questo modo
si può considerare \(\phi = 0\) come assenza di moto, mentre al tendere di
\(\phi\) al valore asintotico \(1\) cresce la confidenza con cui si attesta
che l'utente sia in movimento.

\subsection{Filtro di Kalman}\label{sec:kalman}
Un filtro di Kalman è un particolare filtro ricorsivo che corregge lo stato di
un sistema dinamico attraverso l'analisi di misurazioni soggette a qualche
forma di rumore. Nel caso di questo elaborato, il sistema dinamico è
rappresentato dall'utente che si muove all'interno dell'edificio e le
misurazioni corrispondono all'output del modello di localizzazione indoor
descritto nel capitolo~\ref{cap:architecture}.

Il filtro di Kalman permette di sfruttare le conoscenze del sistema dinamico e
dei parametri di errore del sistema di misurazione in modo da ottenere stime
più precise. A tale scopo si può utilizzare l'euristica prodotta a partire dal
sensore inzeriale e definita precedentemente come \(\phi = \tanh(|\tilde{v}|)\).
Tale coefficiente viene moltiplicato con la stima indiretta della velocità
dell'utente ottenuta attraverso output successivi del modello. Si definisce
quindi la velocità lungo un asse del piano in cui si muove l'utente come:
\[ v = \phi \frac{y_{t+1} - y_t}{\Delta t} \]
dove con \(y_t\) e \(y_{t+1}\) si indica (una componente del) l'output del
modello in due istanti di tempo consecutivi e con \(\Delta t\) l'intervallo di
tempo trascorso tra i due.

Il filtro di Kalman necessita poi di conoscere la varianza del sistema che
emette le misurazioni, la quale si può ottenere dall'ensemble di modelli come
descritto nel paragrafo~\ref{sec:ensemble}. Utilizzando queste informazioni e,
sebbene l'implementazione realizzata sia più un'euristica che un vero filtro di
Kalman, è stato possibile riddure buona parte delle oscillazioni durante la
navigazione.


\end{document}

