\begin{document}

\section{Data Augmentation}

\usemintedstyle{xcode}
\begin{listing}[htp]
	\inputminted[breaklines=true,fontsize=\footnotesize]{python}{listings/augmenter.py}
	\caption{Interfaccia della classe TimeseriesDataAugmenter. La classe si
		occupa di arricchire il dataset di input applicando le trasformazioni
		definite nel Paragrafo~\ref{sec:augmentation}}
\end{listing}

\begin{listing}[htp]
	\inputminted[breaklines=true,fontsize=\footnotesize]{python}{listings/magwarp.py}
	\caption{Implementazione Magnitude Warping}
\end{listing}

\begin{listing}[htp]
	\inputminted[breaklines=true,fontsize=\footnotesize]{python}{listings/shuffle.py}
	\caption{Implementazione Shuffling}
\end{listing}

\begin{listing}[htp]
	\inputminted[breaklines=true,fontsize=\footnotesize]{python}{listings/scale.py}
	\caption{Implementazione Scaling}
\end{listing}

\begin{listing}[htp]
	\inputminted[fontsize=\footnotesize]{python}{listings/replace.py}
	\caption{Implementazione Deattivazione selettiva}
\end{listing}

%%%%%%%%%%%%%%%%%%%%%%%%%%%%%%%%%%%%%%%%%%%%%%%%%%%%%%%%%%%%%%%%%%%%%%%%%%%%%%%
% \section{Rete Neurale}
%%%%%%%%%%%%%%%%%%%%%%%%%%%%%%%%%%%%%%%%%%%%%%%%%%%%%%%%%%%%%%%%%%%%%%%%%%%%%%%
% \section{Applicazione Mobile}

\end{document}
