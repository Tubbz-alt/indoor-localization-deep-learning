\documentclass{standalone} 
\begin{document}
I sistemi di localizzazione indoor, ovvero i sistemi che permettono la
localizzazione di dispositivi all’interno di un ambiente chiuso, dove non è
possibile sfruttare la copertura del sistema GPS, sono stati oggetto di
notevole interesse. Questo elaborato illustra una soluzione tecnologica al
problema della localizzazione indoor basata sull’uso di strumenti di machine
learning. La soluzione proposta sfrutta una rete neurale convoluzionale (CNN)
profonda. I dati di input del modello costituiscono una serie temporale di
segnali broadcast \emph{Bluetooth Low Energy} (BLE) emessi da un insieme di
Beacon disposti all’interno dell’edificio oggetto della localizzazione indoor,
mentre l’output è una coppia di coordinate relative alla posizione all’interno
dell’edificio stesso. La soluzione inoltre utilizza alcune tecniche di
\emph{data augmentation} per generare un dataset di grandi dimensioni sulla
base dei campionamenti dei segnali effettuati in loco.

A seguito dell’addestramento, il modello utilizzato ha mostrato un errore medio
assoluto (MAE) sul dataset di test pari a \emph{30cm}, esibendo una buona
affidabilità anche rispetto a variazioni significative dei segnali dovute al
rumore ambientale. Per ridurre ulteriormente l’errore medio  è stato costruito
un insieme di modelli, ognuno addestrato con diversi iperparametri. Questa
tecnica ha permesso di ridurre l’errore medio fino a circa \emph{26cm}.

Il modello prodotto risulta eseguibile in tempo reale su dispositivi mobile con
ridotte capacità computazionali, rendendolo particolarmente adatto alla
cosiddetta navigazione ``blue-dot'' all’interno di contesti Indoor. Tuttavia le
varie fluttuazioni dell’output del modello tendono ad originare una navigazione
poco fluida.  Per arginare questo problema è stato applicato un filtro di
Kalman al modello e viene sfruttato il sensore inerziale dello smartphone per
produrre un’euristica utile a individuare i movimenti dell’utente.
\end{document}
