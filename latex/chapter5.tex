\begin{document}
La navigazione indoor si è rivelato un problema particolarmente adatto alla
risoluzione tramite tecniche di deep learning, anche se queste ultime non si
sono mostrate esenti da difetti. In particolare la difficoltà nella raccolta
dati per l'addestramento del modello rmiane l'ostacolo più grande in questo
contesto e nel machine learning in generale. Tuttavia l'uso di tecniche di data
augmentation ha mostrato come anche con una quantità iniziale di dati
relativamente scarsa sia possibile generare un modello che stimi la posizione
dell'utente all'interno di un edificio con precisione di circa 30cm.

Le fluttuazioni dell'output del modello durante la navigazione sono state
parzialmente risolte utilizzando un filtro di Kalman, congiuntamente ai dati
del sensore inerziale dello smartphone, ma ulteriori approfondimenti sono
necessari allo scopo di valutare concretamente l'entità di questi
miglioramenti. Infine, ulteriori approcci al problema della stabilizzazione
possono essere esplorati, come discusso nei successivi paragrafi.
\section{Lavori futuri}
Sulla base dei risultati sperimentali ottenuti e sulle difficoltà riscontrate
durante la risoluzione del problema della locallizazione indoor, sono state
individuate varie idee meritevoli di appronfondimento.
\subsection{Particle Filter}
I Particle Filter sono una categoria di algoritmi Monte Carlo per
l'approssimazione di distribuzioni di probabilità non gaussiane. L'utilizzo di
questi metodi si rende utile nel momento in cui il modello è soggetto a rumore
non gaussiano. È verosimile pensare che l'errore commesso dal modello descritto
non presenti una distribuzione gaussiana. In questo contesto, l'utilizzo di un
Particle Filter sarebbe da preferire rispetto ad un filtro di Kalman. In
particolare per il problema della localizzazione indoor, sono stati descritti
vari metodi di tipo Particle Filter che hanno mostrato buoni risultati nella
correzione delle previsioni del sistema di navigazione\cite{particle-filter1,
  particle-filter2}.
\subsection{Input a Lunghezza Variabile}
Nella attuale implementazione della rete neurale, le serie temporali dei
segnali emessi dai beacon hanno dimensione fissa. Ciò significa che la
dimensione del campionamento effettuata durante la fase di navigazione deve
essere sempre la stessa, eventualmente aggiungendo valori fittizi alla serie
temporale. Tuttavia il modello, grazie al livello di Pooling globale, permette
di associare all'input una dimensione qualsiasi lungo l'asse temporale. Sarebbe
quindi possibile addestrare il modello con input di lunghezza variabile e a
tale scopo non sarebbe difficile arricchire il dataset con variazioni dei
segnali originali modificandone la lunghezza. Non è da escludere che un simile
approccio migliori ancora la precisione del modello, oltre a renderlo più
flessibile in fase di utilizzo, non essendo più legato ad una dimensione
costante per l'input dei segnali.
\subsection{Reti Neurali Residuali}
Le reti neurali residuali sono un particolare tipo di architettura,
inizialmente ideato per le reti convoluzionali, che permette di ridurre
notevolmente il problema della scomparsa del gradiente in modelli molto
profondi\cite{resnet}. L'adozione di reti neurali residuali con un numero più
elevato di livelli potrebbe consentire di migliorare la precisione del sistema
di navigazione. Tuttavia l'incremento di profondità del modello andrebbe a
discapito della performance in fase di addestramento e in fase di inferenza.
Rimane quindi dubbia l'utilizzabilità di tali reti in contesti mobile.
\subsection{Variational Autoencoder: Generazione di nuovi dati}
I Variational Autoencoder (VAE) sono modelli di machine learning generativi.
Essi vengono addestrati su un dataset per simulare la generazione di nuovi
esempi, simili a quelli visti durante la procedura di addestramento\cite{vae}.
Utilizzando un VAE è possibile arricchire ulteriormente il dataset di
addestramento del modello proposto in questo elaborato, fornendo esempi nuovi e
possibilmente migliori rispetto a quelli prodotti implementando manualmente
tecniche di data augmentation.
\subsection{Transfer Learning: Input Masking e Ricostruzione dei Segnali}
Un possibile approccio alla risoluzione del problema della raccolta dei dati è
quello che prende il nome di Transfer Learning. Addestrando prima un modello su
una grande quantità di dati privi di target (il valore cioè che il modello
dovrebbe inferire a partire dall'input) su un problema simile all'originale, è
possibile sfruttare la configurazione dei parametri risultante per addestrare
lo stesso modello su un nuovo problema, spesso più complesso, attraverso un
nuovo dataset, stavolta comprendente l'output richiesto. Nel caso della
localizzazione indoor, il problema di tipo non supervisionato potrebbe essere
quello di ricostruire un segnale perturbato con del rumore, o alterato in
qualche modo. Un approccio simile è utilizzato dai modelli di tipo
Transformer\cite{attention}, come BERT\cite{bert} e il più recente
GPT-3\cite{gpt3}, in cui l'input della rete viene mascherato nascondendone una
parte e successivamente ricostruito. Se il modello così addestrato risulta
essere in grado a ricostruire i segnali, è probabile che abbia individuato
delle relazioni tra gli input dei diversi beacon. Ciò dovrebbe aiutare, nel
contesto del problema originale, a costruire una rappresentazione significativa
dell'input anche con una quantità relativamente piccola di dati e quindi
risolvere il problema in modo più efficiente. La raccolta del primo dataset è
effettuabile in modo molto economico navigando l'edificio in cui sono disposti
i beacon e raccogliendo i segnali ricevuti.
% \subsection{Transformers per Problemi di Regressione}
\subsection{Simulatore BLE}
L'utilizzo di un simulatore di propagazione dei segnali bluetooth all'interno
di un edificio potrebbe eliminare completamente la necessità di raccogliere
dati manualmente. L'implementazione di un tale simulatore dovrebbe prevedere
la possibilità di configurare quest'ultimo impostando una planimetria di un
locale e la disposizione dei beacon all'interno di esso. Poichè i segnali
bluetooth emessi sono naturalmente soggetti a rumore ambientale, non dovrebbe
essere necessario modellare la realtà con un elevato grado di precisione.
L'introduzione nel simulatore di sorgenti di rumore casuale dovrebbe essere
sufficiente a rappresentare con la giusta fedeltà il contesto del problema e,
in questo senso, la resistenza al rumore e alle variazioni dell'input del
modello dovrebbero garantire buoni risultati anche con dati sintetici.
\subsection{Posizionamento Magnetico}
La struttura di un edificio e i materiali in esso contenuti alterano il campo
magnetico naturalmente emesso dalla Terra. Queste variazioni sono uniche e
identificabili, rendendo possibile una tecnica di localizzazione indoor che
prende il nome di posizionamento magnetico. Attraverso la mappatura di tutte le
perturbazioni del campo magnetico causate dall'edificio stesso, è possibile
implementare un sistema di navigazione con un discreto grado di
precisione\cite{magnetic-positioning}. Il modello presentato in questo
elaborato sfrutta già le informazioni ricavate dal sensore magnetico dello
smartphone, come descritto nel Paragrafo~\ref{subsec:input}, ma il loro utilizzo
può essere approfondito.
\end{document}
