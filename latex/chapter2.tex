\documentclass{standalone}

\begin{document}
In questo capitolo saranno introdotti i concetti fondamentali alla base delle
moderne tecniche di Deep Learning e le strutture matematiche necessarie alla
loro comprensione. \\

\section{Machine Learning}
Il \emph{Machine Learning}, o apprendimento automatico, è un insieme di
tecniche e algoritmi che consente a dei programmi di ``imparare'' a svolgere
un determinato compito sulla base di esperienze pregresse, senza bisogno da
parte del programmatore di specificare come eseguire tali mansioni. 
\subsection{Regressione Lineare}
Un classico esempio di algoritmo di machine learning è quello della regressione
lineare. Scopo dell'algoritmo è predire l'output di una determinata funzione.
Si consideri quindi un vettore $ \bm x \in \mathbb{R}^n $, e un valore scalare
$\hat{y} = \bm \theta^\intercal \bm x$. il vettore $\bm \theta$ introduce i
parametri del modello, mentre $\hat{y}$ ne rappresenta l'output, che è una
funzione lineare di $\bm x$. Siano quindi $ \bm x_1, \bm x_2, \dotsc, \bm x_m $
dei vettori in $\mathbb{R}^n$ e  $ y_1, y_2, \dotsc, y_m $ i corrispettivi
valori della funzione $f(\bm x_i)$ che stiamo cercando di approssimare.

Perchè la funzione $\hat{y}(\bm x)$ approssimi $f(\bm x)$ è necessario che i
parametri $\bm \theta$ del modello si adattino in modo da minimizzare la differenza
tra l'output prodotto dal modello e la cosiddetta \emph{ground truth}: $y =
f(\bm x)$. A questo scopo si definisce una metrica di errore propria del
processo di apprendimento: l'errore quadratico medio (MSE dall'inglese) 
\begin{equation} \label{eq:mse}
  MSE = \frac{1}{m} \sum_{i=1}^m{(\hat{y} - y_i)^2} 
\end{equation}

Per minimizzare l'errore sul nostro dataset di test è sufficiente porre a zero
la derivata, rispetto a $\bm \theta$, della nostra funzione di costo. Nel caso
della regressione lineare è possibile risolvere l'equazione risultante
ottenendo un sistema di equazioni che prende il nome di 
\emph{normal equations}.
Esistono tuttavia metodi numerici iterativi che si basano sulle informazioni
fornite dal gradiente della funzione di costo che permettono di aggiornare i
parametri del modello cercando di ridurne l'errore, anche nel caso di modelli
non lineari. L'algoritmo su cui si basano molti dei moderni metodi di
apprendimento del Deep Learning è il \emph{gradient descent} o metodo del
gradiente.

Il metodo del gradiente aggiorna i parametri $\bm \theta$ del modello secondo
la seguente regola: 
\begin{equation} \label{eq:grad-up}
  \bm \theta' = \bm \theta - \eta \nabla_{\bm \theta} J(\bm \theta) 
\end{equation}
dove $J(\bm \theta)$ rappresenta la funzione di costo associata al modello,
mentre $\eta$ è un coefficiente chiamato \emph{learning rate}.
%Intuizione gradient descent
Un'interpretazione dell'algoritmo è data dalle informazioni sulla monotonia
ottenute dal gradiente della funzione di costo: per $\eta$ abbastanza piccolo
risulta $J(\bm \theta') \leq J(\bm \theta)$ poichè il gradiente negativo di $J$
determina la direzione di massima decrescita della funzione\cite{goodfellow}.
Ne consegue che applicando ricorsivamente la regola di aggiornamento del metodo
del gradiente, la nostra funzione $\hat{y}$ tenderà ad avvicinarsi alla
funzione originale $y$, minimizzando la funzione di costo.
Possiamo intepretare il coefficiente $\eta$ come la velocità con cui seguire la
pendenza della funzione di errore. \\
{\large TODO: Inserire un'immagine che fornisca un'intuizione del
  funzionamento del gradient descent}

%Gradient descent con MSE e modello lineare
Nel caso di un modello di regressione lineare che usa l'MSE come funzione di
costo, risulta: 
\begin{align*}
  & \forall j \in {1, \dotsc, n}: \\
  \frac{\partial}{\partial\theta_j} MSE &= 
  \frac{\partial}{\partial\theta_j} \frac{1}{m} \sum_{i=1}^m{(\hat{y}_i -
    y_i)^2} \\ 
  %
  &= \frac{\partial}{\partial\theta_j} \frac{1}{m}
  \sum_{i=1}^m{(\bm \theta^\intercal \bm x_i - y_i)^2} \\
  %
  &= \frac{1}{m}
  \sum_{i=1}^m{\frac{\partial}{\partial\theta_j}(\bm \theta^\intercal \bm x_i -
    y_i)^2} \\
  %
  &= \frac{2}{m}
  \sum_{i=1}^m{(\bm \theta^\intercal \bm x_i - y_i)x_i^{(j)}}
\end{align*}
%
Ovvero abbiamo che la regola di aggiornamento è:
$$ \theta'_j = \theta_j - \eta \frac{\partial}{\partial\theta_j} J(\bm \theta)
             = \theta_j - \eta (\frac{2}{m} 
             \sum_{i=1}^m{(\bm \theta^\intercal \bm x_i - y_i)x_i^{(j)}}),
             \quad \forall j \in \{1, \dotsc, n\}
$$

\section{Perceptron}
Il \emph{Perceptron} è il modello che ha posto le basi per le moderne reti
neurali artificiali e il deep learning. Esso prende spunto dalla neurologia,
cercando di imitare il comportamento dei neuroni del cervello umano, con ovvie
limitazioni e senza presunzione di volerne fornire una simulazione accurata del
funzionamento. Una schematizzazione del modello è descritta in Figura
\ref{fig:perceptron}
\def\nodesize{1.0cm}
\tikzstyle{neuron}=[circle, draw=black, minimum size=\nodesize]
\tikzstyle{dots}=[draw=none, scale=2, text height=0.333cm, execute at end node=$\vdots$]
\centering
\begin{tikzpicture}[x=1.5cm, y=1.5cm, >=stealth] % >=stealth arrow style
  % input nodes
  \foreach \m [count=\y] in {1,2,3,dots,n} 
  {
    \ifnum\y=4
    \node [\m/.try] at (0,-\y) {};
    \else
    \node [neuron] (input-\m) at (0,-\y) {$x_\m$};
    \fi
  }
  % middle node
  \node [neuron] (middle) at (1.5, -3) {\LARGE $\Sigma$};

  % activation function
  \node [neuron] (activation) at (3.0, -3) {$g$};

  % connections
  \foreach \m [count=\y] in {1,2,3,n}
  % above moves the text higher depending on the orientation of the line
  \draw [thick, ->] (input-\m) -- (middle) 
  node[above=4pt*abs(3 - \y), pos=0.45, scale=0.9] {$\theta_\m$};

  % middle point between (middle) and (activation) nodes
  \path [thick, ->, draw] (middle) -- (activation) coordinate[pos=0.5] (activation-path);


  % draw sigmoid symbol betwwen (middle) and (activation)
  \draw [thick, scale=0.3, color=funcolor, x=5, y=70, 
  domain=-6:6, yrange=-1:1, 
  shift={($ (activation-path) - (0, 0.5) $) }] % calc library 
  plot (\x, {1/(exp(-\x)+1)});

  % alternative representation of the sigmoid symbol using an integral
  % \draw [thick, ->] (middle) -- (activation) node[midway, above] {
  %   $\displaystyle \int$
  % };

\end{tikzpicture}


Il perceptron è molto simile al modello di regressione lineare descritto
precedentemente, ma si distingue per un fattore fondamentale: la non linearità.
L'output del perceptron è infatti definito come:
$$\hat{y}(\bm x) = g(\bm \theta^\intercal \bm x)$$
dove $g$ è una funzione non lineare che, in questo caso, è una funzione
\emph{sigmoidea}, cioè una funzione che ha un andamento a ``S'', con due
asintoti orizzontali come in Figura \ref{fig:sigmoid}: di solito è utilizzata
la funzione $\displaystyle g(x) = \frac{1}{1 + e^{-x}}$.

Malgrado la nonlinearità del modello, il perceptron non è in grado di
approssimare molte classi di funzioni. È famoso l'esempio della funzione XOR,
la quale non può essere imparata dal perceptron. Per questo motivo è stata
sviluppata un'estensione del modello che prende il nome di \emph{Multi Layer
  Perceptron}.

\begin{figure}[!htp] % place it here (h)
  \centering
  \begin{tikzpicture}
    \begin{axis}[
      grid=both,
      grid style={line width=.1pt, draw=gray!30},
      xlabel=$x$, ylabel={$g(x)$}
      ]
      \addplot[very thick, color=funcolor]{1/(1 + exp(x)^(-1))};
    \end{axis}
  \end{tikzpicture}
  \caption{Funzione sigmoidea $\displaystyle g(x) = \frac{1}{1 + e^{-x}}$ }
  \label{fig:sigmoid}
\end{figure}


\end{document}

