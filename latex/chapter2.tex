\documentclass{standalone}

\begin{document}
In questo capitolo saranno introdotti i concetti fondamentali alla base delle
moderne tecniche di Deep Learning e le strutture matematiche necessarie alla
loro comprensione. \\

\section{Machine Learning}
Il \emph{Machine Learning}, o apprendimento automatico, è un insieme di
tecniche e algoritmi che consentono a dei programmi di ``imparare'' a svolgere
un determinato compito sulla base di esperienze pregresse, senza bisogno da
parte del programmatore di specificare come eseguire tale mansione. 
\subsection{Regressione Lineare}
Un esempio classico di algoritmo di machine learning è quello della regressione
lineare. Scopo dell'algoritmo è quello di predire l'output di una determinata
funzione.  Si consideri quindi un vettore $ \bm x \in \mathbb{R}^n $, e
un valore scalare $\hat{y} = \bm \theta^\intercal \bm x$. il
vettore $\bm \theta$ introduce i parametri del modello, mentre
$\hat{y}$ ne rappresenta l'output, che è una funzione lineare di $\bm
x$. Siano quindi $ \bm x_1, \bm x_2, \dotsc, \bm x_m $ dei vettori in
$\mathbb{R}^n$ e  $ y_1, y_2, \dotsc, y_m $ i corrispettivi valori della
funzione $f(\bm x_i)$ che stiamo cercando di approssimare.

Perchè la funzione $\hat{y}(\bm x)$ approssimi $f(\bm x)$ è necessario che i
parametri $\bm \theta$ del modello si adattino in modo da minimizzare la differenza
tra l'output prodotto dal modello e la cosiddetta \emph{ground truth}: $y =
f(\bm x)$. A questo scopo si definisce una metrica di errore propria del
processo di apprendimento: l'errore quadratico medio (MSE dall'inglese) 
$$MSE = \frac{1}{m} \sum_{i=1}^m{(\hat{y} - y_i)^2}$$

Per minimizzare l'errore sul nostro dataset di test è sufficiente porre a zero
la derivata, rispetto a $\bm \theta$, della nostra funzione di costo. Nel caso
della regressione lineare è possibile risolvere l'equazione risultante
ottenendo un sistema di equazioni che prende il nome di 
\emph{normal equations}.
Esistono tuttavia metodi numerici iterativi che si basano sulle informazioni
fornite dal gradiente della funzione di costo che permettono di aggiornare i
parametri del modello cercando di ridurne l'errore, anche nel caso di modelli
non lineari. L'algoritmo su cui si basano molti dei moderni metodi di
apprendimento del Deep Learning è il \emph{gradient descent} o metodo del
gradiente.

Il metodo del gradiente aggiorna i parametri $\bm \theta$ del modello secondo
la seguente regola: 
$$ \bm \theta' = \bm \theta - \eta \nabla_{\bm \theta} J(\bm \theta) $$
dove $J(\bm \theta)$ rappresenta la funzione di costo associata al modello,
mentre $\eta$ è un coefficiente chiamato \emph{learning rate}.
%Intuizione gradient descent
Un'interpretazione dell'algoritmo è data dalle informazioni sulla monotonia
ottenute dal gradiente della funzione di costo: per $\eta$ abbastanzza piccolo
risulta $J(\bm \theta') \leq J(\bm \theta)$ poichè il gradiente negativo di $J$
determina la direzione di massima decrescita della funzione\cite{goodfellow}.
Ne consegue che applicando ricorsivamente la regola di aggiornamento del metodo
del gradiente, la nostra funzione $\hat{y}$ tenderà ad avvicinarsi alla
funzione originale $y$, minimizzando la funzione di costo.
Possiamo intepretare il coefficiente $\eta$ come la velocità con cui seguire la
pendenza della funzione di errore.

%Gradient descent con MSE e modello lineare
Nel caso di un modello di regressione lineare che usa l'MSE come funzione di
costo, risulta: 
\begin{align*}
  &\forall j \in {1, \dotsc, n}: \\
  %
  \frac{\partial}{\partial\theta_j} MSE
  &= 
  \frac{\partial}{\partial\theta_j} \frac{1}{m} \sum_{i=1}^m{(\hat{y}_i -
    y_i)^2} \\ 
  %
  &= \frac{\partial}{\partial\theta_j} \frac{1}{m}
  \sum_{i=1}^m{(\bm \theta^\intercal \bm x_i - y_i)^2} \\
  %
  &= \frac{1}{m}
  \sum_{i=1}^m{\frac{\partial}{\partial\theta_j}(\bm \theta^\intercal \bm x_i -
    y_i)^2} \\
  %
  &= \frac{2}{m}
  \sum_{i=1}^m{(\bm \theta^\intercal \bm x_i - y_i)x_i^{(j)}}
\end{align*}
%
Ovvero abbiamo che la regola di aggiornamento è:
$$ \theta'_j = \theta_j - \eta \frac{\partial}{\partial\theta_j} J(\bm \theta)
             = \theta_j - \eta (\frac{2}{m} 
             \sum_{i=1}^m{(\bm \theta^\intercal \bm x_i - y_i)x_i^{(j)}}),
             \quad \forall j \in \{1, \dotsc, n\}
$$




\section{Multi Layer Perceptron}




\end{document}

